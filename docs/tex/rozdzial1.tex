\newpage
\section{Ogólne określenie wymagań}		%1

\hspace{0.60cm}
W ramach realizacji projektu zostały zdefiniowane następujące wymagania dotyczące struktury, funkcjonalności oraz środowiska pracy aplikacji implementującej listę dwukierunkową:

\subsection{Projekt wieloplikowy}
Projekt powinien być zorganizowany w sposób modularny, z rozdzieleniem odpowiedzialności na różne pliki źródłowe i nagłówkowe. W szczególności:

\begin{itemize}
	\item Kod implementujący logikę listy dwukierunkowej powinien być umieszczony \\ w osobnym pliku źródłowym (\texttt{doubly\_linked\_list.cpp}) oraz odpowiadającym mu pliku nagłówkowym (\texttt{doubly\_linked\_list.hpp}).
	\item Funkcje obsługujące interfejs aplikacji, takie jak menu oraz obsługa interakcji z użytkownikiem, powinny być umieszczone w dedykowanym pliku (\texttt{app.cpp}).
\end{itemize}

\subsection{Użycie systemu kontroli wersji Git i platformy GitHub}
Projekt powinien być zarządzany za pomocą systemu kontroli wersji Git, a repozytorium projektu powinno być umieszczone na platformie GitHub, co pozwoli na:

\begin{itemize}
	\item Śledzenie zmian w kodzie i historii projektu.
	\item Współpracę z innymi członkami zespołu.
	\item Zastosowanie mechanizmu Continuous Integration (CI) w celu automatycznej weryfikacji poprawności kodu.
\end{itemize}

\subsection{Dokumentacja projektowa}
Dokumentacja projektu powinna być dostępna w dwóch formach:

\begin{itemize}
	\item \textbf{Dokumentacja zautomatyzowana} – generowana automatycznie z kodu za pomocą narzędzia \texttt{Doxygen}. Powinna zawierać szczegółowe informacje o strukturze klas, funkcjach oraz parametrach.
	\item \textbf{Dokumentacja teoretyczna} – opracowana ręcznie w formacie \LaTeX{}, obejmująca między innymi opis wymagań, architekturę systemu oraz instrukcję użytkownika.
\end{itemize}

\subsection{Funkcjonalność listy dwukierunkowej}
Implementacja listy dwukierunkowej powinna oferować następujące funkcje:

\begin{itemize}
	\item Dodawanie elementów na początku i końcu listy.
	\item Wstawianie elementów na dowolnym indeksie.
	\item Usuwanie elementów z początku, końca oraz z wybranego indeksu.
	\item Wyświetlanie wszystkich elementów listy w porządku od początku do końca oraz w odwrotnej kolejności.
	\item Usuwanie wszystkich elementów z listy.
\end{itemize}

\subsection{Interfejs CLI użytkownika}
Aplikacja powinna oferować interfejs linii poleceń (CLI) umożliwiający użytkownikowi korzystanie z funkcji listy w sposób intuicyjny. Interfejs powinien obejmować:

\begin{itemize}
	\item Wyświetlanie menu wyboru operacji.
	\item Obsługę dodawania, usuwania i wyświetlania elementów.
	\item Możliwość przeglądania listy w obu kierunkach, zarówno od początku do końca, jak i w odwrotnej kolejności.
	\item Opcję zakończenia pracy z aplikacją.
\end{itemize}

\subsection{Obsługa platform}
Aplikacja powinna działać poprawnie na następujących platformach:

\begin{itemize}
	\item \textbf{Windows} – wsparcie dla kompilacji i uruchomienia na systemach z rodziny Windows.
	\item \textbf{Linux} – wsparcie dla kompilacji i uruchomienia na systemach operacyjnych \\ z rodziny Linux.
\end{itemize}
