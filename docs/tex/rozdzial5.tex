\newpage
\section{Wnioski}	%5

Podsumowanie projektu

Przeprowadzony projekt dotyczący implementacji listy dwukierunkowej przyniósł szereg istotnych rezultatów i wniosków. Zrealizowane zadania obejmowały analizę problemu, projektowanie, implementację oraz testowanie funkcjonalności algorytmu. Dzięki temu udało się stworzyć stabilną i wydajną strukturę danych, która może być używana w różnych zastosowaniach programistycznych.

\subsection{Wnioski dotyczące implementacji}

\begin{enumerate}
	\item \textbf{Efektywność operacji}: Implementacja podwójnie powiązanej listy umożliwia szybkie dodawanie i usuwanie elementów z dowolnych miejsc w strukturze. Dzięki zastosowaniu wskaźników do poprzednich i następnych węzłów, operacje te są wykonywane w czasie stałym O(1).
	\item \textbf{Łatwość rozbudowy}: Struktura ta może być łatwo rozszerzana o dodatkowe funkcjonalności, takie jak wyszukiwanie czy sortowanie elementów. Modularność kodu umożliwia przyszłe modyfikacje oraz dostosowanie do nowych wymagań.
	\item \textbf{Wykorzystanie narzędzi}: W projekcie zastosowano szereg narzędzi, takich jak Git i GitHub do zarządzania wersjami, a także Doxygen do automatycznego generowania dokumentacji. Narzędzia te znacznie ułatwiły współpracę i organizację pracy nad projektem.
	\item \textbf{Testowanie i weryfikacja}: Przeprowadzone testy ujawniły, że implementacja działa zgodnie z założeniami. Każda z operacji na liście została dokładnie przetestowana, co potwierdziło jej niezawodność.
	\item \textbf{Zastosowanie w praktyce}: Struktura danych w postaci podwójnie powiązanej listy może być wykorzystana w wielu aplikacjach, takich jak edytory tekstu, aplikacje do zarządzania zadaniami, czy systemy bazodanowe. Jej elastyczność i wydajność czynią ją atrakcyjnym rozwiązaniem w projektach programistycznych.
\end{enumerate}

\newpage

\subsection{Perspektywy rozwoju}

Na cele tego projektu utwrzono szablon (pod adresem pod adresem: \url{https://github.com/Me-Phew/programowanie-zaawansowane-template}\cite{GitHubProjectTemplate} (Dostęp: 24.10.2024r.)), który stanowi solidną bazę dla przyszłych projektów.
Repozytorium z kodem projektu znajduje się pod adresem \url{https://github.com/Me-Phew/doubly-linked-list}\cite{GitHubProject}
